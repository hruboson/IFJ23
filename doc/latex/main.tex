\documentclass[12pt]{article}

\usepackage{siunitx}
\usepackage{enumerate}

\usepackage{subfiles} % Best loaded last in the preamble

\begin{document}

\title{Implementace překladače imperativního jazyka IFJ23}
\author{
    Tým xhrubo01, varianta TRP-izp
    \and
    % abecedně
    Dominik Borek (xborek12)
	\and
    Ondřej Hruboš (xhrubo01)
	\and
	Radek Jestřabík (xjestr04)
	\and
	Ondřej Šatinský (xsatin03)
}
%\affil{Vysoké učení technické v Brně - Fakulta informačních technologií}
\date{\today}

\maketitle

\begin{abstract}
	// todo
\end{abstract}

\section{Rozdělení práce}
// todo

%=============================================================================%
% ZADÁNÍ
%=============================================================================%
%V dokumentaci popisujte návrh (části překladače a předávání informací mezi nimi),
%implementaci (použité datové struktury, tabulku symbolů, generování kódu), vývojový cyk-
%lus, způsob práce v týmu, speciální použité techniky a algoritmy a různé odchylky od před-
%nášené látky či tradičních přístupů. Nezapomínejte také citovat literaturu a uvádět refe-
%rence na čerpané zdroje včetně správné citace převzatých částí (obrázky, magické konstanty,
%vzorce). Nepopisujte záležitosti obecně známé či přednášené na naší fakultě.
%Dokumentace musí povinně obsahovat (povinné tabulky a diagramy se nezapočítá-
%vají do doporučeného rozsahu):
%• 1. strana: jména, příjmení a přihlašovací jména řešitelů (označení vedoucího) + údaje
%o rozdělení bodů, identifikaci vaší varianty zadání ve tvaru “Tým login_vedoucího,
%varianta 𝑋” a výčet identifikátorů implementovaných rozšíření.
%• Rozdělení práce mezi členy týmu (uveďte kdo a jak se podílel na jednotlivých částech
%projektu; povinně zdůvodněte odchylky od rovnoměrného rozdělení bodů).
%• Diagram konečného automatu, který specifikuje lexikální analyzátor.
%• LL-gramatiku, LL-tabulku a precedenční tabulku, podle kterých jste implementovali
%váš syntaktický analyzátor.
%• Stručný popis členění implementačního řešení včetně názvů souborů, kde jsou jed-
%notlivé části včetně povinných implementovaných metod překladače k nalezení

% ZATÍM POUŽITY PRACOVNÍ NÁZVY PRO SEKCE
\section{Scanner}
\subfile{sections/scanner.tex}

\section{Parser}
\subfile{sections/parser.tex}

\section{Semantic}
\subfile{sections/semantic.tex}

\section{Vnitřní kód}
\subfile{sections/ir.tex}

\section{Generátor cílového kódu}
\subfile{sections/code_generator.tex}

\end{document}