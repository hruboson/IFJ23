\documentclass[12pt]{article}

\usepackage[utf8]{inputenc}
\usepackage[left=2cm, top=3cm, text={17cm, 24cm}]{geometry}
\usepackage{times}
\usepackage{verbatim}
\usepackage{enumitem}
\usepackage{graphicx} % vkládání obrázků
\usepackage{subfiles}


\begin{document}

    \begin{titlepage}
        \begin{center}
            \includegraphics[width=0.77\linewidth]{img/fit_logo.png} \\
            \vspace{\stretch{0.4}}        
            \Huge{Projektová dokumentace} \\
            \Large{\textbf{Implementace překladače imperativního jazyka IFJ23}} \\
            \Large{Tým xhrubo01, varianta TRP-izp} \\ 
            \vspace{\stretch{0.6}}
        \end{center}
        \begin{minipage}{0.4 \textwidth}
			{\Large \today}
		\end{minipage}
		\hfill
		\begin{minipage}[r]{0.6 \textwidth}
			\normalsize
            \begin{flushright}
                \begin{tabular}{l l l}
    				Dominik Borek & (xborek12)\\
    				\textbf{Ondřej Hruboš} & \textbf{(xhrubo01)} \\
    				Radek Jestřabík & (xjestr04)  \\
    				Ondřej Šatinský & (xsatin03)  \\
    			\end{tabular}
            \end{flushright}
		\end{minipage}
    \end{titlepage}

\begin{abstract}
	// todo
\end{abstract}

\section{Rozdělení práce}
// todo

%=============================================================================%
% ZADÁNÍ
%=============================================================================%
%V dokumentaci popisujte návrh (části překladače a předávání informací mezi nimi),
%implementaci (použité datové struktury, tabulku symbolů, generování kódu), vývojový cyk-
%lus, způsob práce v týmu, speciální použité techniky a algoritmy a různé odchylky od před-
%nášené látky či tradičních přístupů. Nezapomínejte také citovat literaturu a uvádět refe-
%rence na čerpané zdroje včetně správné citace převzatých částí (obrázky, magické konstanty,
%vzorce). Nepopisujte záležitosti obecně známé či přednášené na naší fakultě.
%Dokumentace musí povinně obsahovat (povinné tabulky a diagramy se nezapočítá-
%vají do doporučeného rozsahu):
%• 1. strana: jména, příjmení a přihlašovací jména řešitelů (označení vedoucího) + údaje
%o rozdělení bodů, identifikaci vaší varianty zadání ve tvaru “Tým login_vedoucího,
%varianta 𝑋” a výčet identifikátorů implementovaných rozšíření.
%• Rozdělení práce mezi členy týmu (uveďte kdo a jak se podílel na jednotlivých částech
%projektu; povinně zdůvodněte odchylky od rovnoměrného rozdělení bodů).
%• Diagram konečného automatu, který specifikuje lexikální analyzátor.
%• LL-gramatiku, LL-tabulku a precedenční tabulku, podle kterých jste implementovali
%váš syntaktický analyzátor.
%• Stručný popis členění implementačního řešení včetně názvů souborů, kde jsou jed-
%notlivé části včetně povinných implementovaných metod překladače k nalezení

% ZATÍM POUŽITY PRACOVNÍ NÁZVY PRO SEKCE
\section{Scanner}
\subfile{sections/scanner.tex}

\section{Parser}
\subfile{sections/parser.tex}

\section{Semantic}
\subfile{sections/semantic.tex}

\section{Vnitřní kód}
\subfile{sections/ir.tex}

\section{Generátor cílového kódu}
\subfile{sections/code_generator.tex}

\end{document}